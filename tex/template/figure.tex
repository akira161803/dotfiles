\usepackage[dvipdfmx]{graphicx}


\begin{figure}[htbp]
  \centering
  \includegraphics[width=\textwidth]{figure/00I.eps} 
  \caption{Accepted Rateの解析結果} 
  \label{fig:results}
\end{figure}

%
%\textwidth: 本文領域(左右の余白を除いた部分)の横幅。 
%\columnwidth: 列の横幅。2段組の論文なら「1段分」の幅になります。 
%\linewidth: 現在の行の横幅。箇条書き(itemize)の中などで使うと、インデント分を差し引いた幅に合わせてくれます。
%

%横に並べる
\begin{figure}[htbp]
  \centering
  \includegraphics[width=0.45\textwidth]{figure1.eps}
  \hspace{10pt} % 横の隙間
  \includegraphics[width=0.45\textwidth]{figure2.eps}
  \caption{左右に並べた画像}
\end{figure}

%subcationを使う場合
\usepackage{subcaption}
\begin{figure}[htbp]
  \centering
  \begin{subfigure}{0.45\textwidth}
    \includegraphics[width=\textwidth]{fig_a.eps}
    \caption{Iフレームの結果}
  \end{subfigure}
  \hfill %横の場合
  \vspace{10pt} %縦の場合
  \begin{subfigure}{0.45\textwidth}
    \includegraphics[width=\textwidth]{fig_b.eps}
    \caption{Pフレームの結果}
  \end{subfigure}
  \caption{全体の解析結果の比較}
\end{figure}
