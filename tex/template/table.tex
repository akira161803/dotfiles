\usepackage{float}
\usepackage{booktabs}


\begin{table}[H]
\centering
\caption{table name}
\begin{tabular}{lccc}
\toprule
フレームタイプ & 前方参照 & 後方参照 & 例(GOP内) \\
\midrule
I フレーム & なし & なし & I(0): なし \\
P フレーム & 直前のI/P & なし & P(3): I(0), P(6): P(3) \\
B フレーム & 直前のI/P & 直後のI/P & B(1): I(0)→P(3) \\
\bottomrule
\end{tabular}
\end{table}

%l左揃え, r右揃え, c中央, p{4cm}(幅4cmの段落形式)
%\begin{tabular}{lccc|cc|r|c p{3cm}|c|}

% \toprule、\midrule、\bottomrule
%(booktabsパッケージ)を使うと、線が少し太くなり、上下に適切なスペース
%が空くため、 より見栄えの良い表が作成できます。

\begin{table}[htbp]
  \centering
  \caption{シミュレーション条件一覧}
  \begin{tabular}{|l|c|r|} \hline
    項目 & 設定値 & 備考 \\ \hline
    距離 & 10m & 近接環境 \\ \hline
    フレームレート & 30fps & 固定値 \\ \hline
    MCS & MCS8 & 256QAM \\ \hline
  \end{tabular}
\end{table}
