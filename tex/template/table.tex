\usepackage{booktabs}

\begin{table}[H]
\centering
\begin{tabular}{lccc}
\toprule
フレームタイプ & 前方参照 & 後方参照 & 例(GOP内) \\
\midrule
I フレーム & なし & なし & I(0): なし \\
P フレーム & 直前のI/P & なし & P(3): I(0), P(6): P(3) \\
B フレーム & 直前のI/P & 直後のI/P & B(1): I(0)→P(3) \\
\bottomrule
\end{tabular}
\caption{table name}
\end{table}

%l左揃え, r右揃え, c中央, p{4cm}(幅4cmの段落形式)
%\begin{tabular}{lccc|cc|r|c p{3cm}|c|}

%$\text{\textbackslash hline}$ の代わりに \toprule、\midrule、\bottomrule
%(booktabsパッケージ)を使うと、線が少し太くなり、上下に適切なスペースが空くため、
%より見栄えの良い表が作成できます。
