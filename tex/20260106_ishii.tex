\documentclass[12pt,dvipdfmx,a4paper]{jsarticle}

\usepackage{amsmath}
\usepackage{mdframed}
\usepackage{xcolor}
\usepackage{booktabs}
\usepackage[hidelinks]{hyperref} 

% 本文
\url{https://sites.google.com/view/gemini} % URLをそのまま表示
\href{https://sites.google.com/view/gemini}{Geminiのサイト} % 好きな文字にリンクを貼る

\title{ミーティング資料}
\author{B3 AJG23066 石井章}
\date{\today}

\begin{document}
\maketitle

\section{目的}
\begin{itemize}
    \item 画像フレーム単位での許容遅延以内での受信側到達率およびロス率の評価.
    \item A-MPDU を画像フレーム単位で適用した場合の性能向上効果の分析.
    \item EDCA を用いて I フレームを優先送信した場合の性能比較.
    \item 可能であれば OFDMA 伝送適用時の性能向上効果の評価.
\end{itemize}

\section{進捗}
\begin{itemize}
    \item ns-3 のインストール
\end{itemize}


\subsection{シミュレーション環境とパラメータ設定}

シミュレーションシステムの基本設定を表\ref{tab:sim_params}に示す.

\begin{table}[H]
\centering
\caption{シミュレーション基本パラメータ}
\begin{tabular}{cc}
\toprule
パラメータ & 値 \\
\midrule
シミュレータ & ns-3.46.1 \\
シミュレーション時間 & 10 秒 \\
フロー & 下り方向 \\
有線のデータレート & 1Gbps \\
有線遅延 & 1ms \\
無線標準 & 802.11be (Wi-Fi 7) \\
周波数帯域 & 2.4 GHz, 20 MHz 幅 \\
レート制御 & MCS8 \\
AP-STA間距離 & 10m \\
UDPペイロードサイズ & 1400 バイト \\
フレームレート & 30 fps (33.3 ms 間隔) \\
I, P, Bパケット数 & それぞれ50, 15, 5 \\
バースト送信時のパケット送信間隔 & 10\(\mu\)s \\
\bottomrule
\end{tabular}
\label{tab:sim_params}
\end{table}


\subsection{画像フレームのパケット数の設定}
ffprobeというツールを使用してmp4動画のI, P, B
フレームのデータ量を調べ、それをパケットサイズの
1400byteで割ることで決定した。

\begin{mdframed}[backgroundcolor=gray!10, hidealllines=true]
\begin{verbatim}
 $ ffprobe -v error -select_streams v:0 -show_frames 
 -show_entries frame=pict_type,pkt_size -of csv 
 video.mp4 
 | head -5
frame,146191,I
frame,5688,B
frame,32288,P
frame,5236,B
frame,37237,P
\end{verbatim}
\end{mdframed}

これによって得られる結果をI, P, Bフレームごとに
ファイルに分け、フレームサイズの平均を求めた。

\begin{mdframed}[backgroundcolor=gray!10, hidealllines=true]
\begin{verbatim}
python3 avg.py Iframe.csv
解析ファイル: Iframe.csv
対象フレーム数: 36
平均データ量: 83054.28 bytes
パケット数: 83054.28 / 1400 = 59.32
\end{verbatim}
\end{mdframed}

サンプルとして以下の三つの動画を使用した(すべて1080p)。
それぞれミュージックビデオ、アニメ、スポーツの代表として想定している。
\begin{itemize}
  \item \href{https://www.youtube.com/watch?v=WJzSBLCaKc8&list=RDWJzSBLCaKc8&start_radio=1}{ヒカキンyoutubeテーマソング} 
  \item \href{https://www.youtube.com/watch?v=kVHimB7_cjQ}{ジョジョの奇妙な冒険予告} 
  \item \href{https://www.youtube.com/watch?v=8X_f465Wli8}{井上尚弥vsピカソ戦ハイライト} 
\end{itemize}

それぞれの動画のI, P, Bフレームをパケット数に換算した結果は以下の通り.
\begin{table}[htbp]
  \centering
  \caption{画像フレームの平均パケット数}
  \begin{tabular}{c|ccc}
    & I & P & B \\ 
    ヒカキン & 34.72 & 13.13 & 5.04 \\ 
    ジョジョ & 47.95 & 15.20 & 4.34 \\
    井上尚弥 & 59.32 & 20.04 & 7.55 \\
  \end{tabular}
\end{table}

動画によって画像フレームのデータ量にばらつきがあることがわかった。
本実験では妥当な値として、
Iフレーム=50パケット、Pフレーム=15パケット、Bフレーム=5パケット
とすることにした。


\section{次回までの目標}
\begin{itemize}
  \item ns3のさらなる理解
\end{itemize}

\end{document}
